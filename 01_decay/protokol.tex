\documentclass{protokol}

\usepackage[czech]{babel}
\usepackage[utf8]{inputenc}
\usepackage{icomma}

% Plovouci bloky (obrazky, tabulky)
\usepackage{floatrow}
\floatsetup[table]{capposition=top}
\floatsetup[figure]{frameset={\fboxsep16pt}}
\usepackage{subcaption}

% Tabulky
\usepackage{tabu}
\usepackage{booktabs}
\usepackage{csvsimple}
\usepackage{multirow}
\usepackage{multicol}
\usepackage{pgfplotstable}
\pgfplotstableset{
	use comma,
	set thousands separator = {},
	every head row/.style = {
		before row = \toprule,
		after row = \midrule
	},
	every last row/.style = {after row = \bottomrule},
}

% Jednotky
\usepackage{siunitx}
\sisetup{
	locale               = DE,
	group-digits         = false,
	inter-unit-product   = \ensuremath{{}\cdot{}},
	list-units           = single,
	list-separator       = {; },
	list-final-separator = \text{ a },
	list-pair-separator  = \text{ a },
	range-phrase         = \text{ až },
	range-units          = single,
	per-mode             = symbol,
}
\usepackage{amsmath}

% Obvody
\usepackage{circuitikz}

% Obrazky a grafy
% \usepackage{graphicx}
\graphicspath{
	{plots/}
	{build/plots/}
}
\usepackage{epstopdf}
\epstopdfsetup{outdir=./build/plots/}
\pgfplotsset{compat=1.17}

\usepackage[hidelinks,pdfusetitle]{hyperref}

\jmenopraktika={Praktikum z fyziky plazmatu}  % jmeno predmetu
\jmeno={Pavel Kosík, Jan Slaný}                             % jmeno mericiho
\obor={F}                               % zkratka studovaneho oboru
\skupina={Út 15:00}                     % doba vyuky seminarni skupiny
\rocnik={IV}
\semestr={VIII}

\cisloulohy={01}
\jmenoulohy={Rozpad plazmatu}

\datum={1. března 2022}                  % datum mereni ulohy
\tlak={}% [hPa]
\teplota={}% [C]
\vlhkost={}% [%]

\newcommand\euler{e}
\newcommand\emass{m_\mathrm{e}}
\newcommand\elemcharge{e}
\newcommand\permitvac{\varepsilon_0}

\newcommand\tm{t}
\newcommand\freq{f}
\newcommand\freqbase{\freq_0}
\newcommand\freqplas{\freq_1}
\newcommand\freqsig{\freq'}
\newcommand\deltafreq{\Delta\freq}
\newcommand\curdis{I}
\newcommand\dens{n}
\newcommand\densinit{n_0}
\newcommand\pres{p}
\newcommand\diffuse{D}
\newcommand\diffuseunit{\metre\squared\per\second}
\newcommand\recomb{\alpha}
\newcommand\recombunit{\metre\cubed\per\second}
\newcommand\diffuselen{\Lambda}
\newcommand\doll{\diffuse/\diffuselen^2}
\newcommand\dollunit{\second^{-1}}
\newcommand\tuberadius{r_\mathrm{t}}
\newcommand\resradius{r_\mathrm{r}}

\begin{document}
\header

\section{Teorie}
Po přerušení přívodu energie do plazmatu se plazma začne rozpadat.
Rozpadem rozumíme postupné klesání koncentrace nabitých částic,
které je způsobeno převážně objemovou rekombinací a~difuzí,
při níž elektrony difundují na stěny výbojky a rekombinují.

Úkolem tohoto praktika je zkoumat konkrétní výboj a~určit,
který z~procesů při jeho rozpadu převládá.
K~tomu je nutno stanovit koeficient difuze $\diffuse$
a~koeficient objemové rekombinace $\recomb$.

\subsection{Difuze v plazmatu}
Při nižších tlacích je převládajícím mechanismem rozpadu difuze.
Pro popis difuzních ztrát vyjdeme z~rovnice kontinuity elektronů:
\begin{equation}
	\label{eq:continuity}
	\frac{\partial \dens}{\partial \tm} - \nabla^2 (\diffuse\dens) = 0,
\end{equation}
kde $\dens$ je koncentrace elektronů a $\diffuse$ je koeficient difuze.
Za předpokladu, že $\dens$ je konstantní v~celém objemu,
výbojka má válcový tvar a že difuze probíhá výhradně v~základním vidu,
můžeme koncentraci elektronů v~čase vyjádřit jako:
\begin{equation}
	\label{eq:dens-diffuse}
	\dens(\tm)= \densinit\euler^{-\frac{\diffuse}{\diffuselen^2}\tm},
\end{equation}
kde $\densinit = \dens(0)$ je počáteční koncentrace
a~$\diffuselen$ je difuzní délka.
Její hodnota je přibližně $\diffuselen \approx \frac{\tuberadius}{\num{2.405}}$,
kde $\tuberadius$ je poloměr trubice a~faktor $\num{2.405}$ je první kořen
Besselovy funkce prvního druhu $J_0$.

\subsection{Objemová rekombinace}
S~rostoucím tlakem roste vliv rekombinace nabitých částic v~objemu plazmatu.
Rekombinační ztráty lze popsat jednoduchou diferenciální rovnicí:
\begin{equation}
	\label{eq:dens-recomb}
	\frac{\mathrm{d} n}{\mathrm{d} t} = -\recomb \dens^2,
\end{equation}
kde $\recomb$ je koeficient rekombinace.
Řešením této rovnice je časový vývoj ve~tvaru:
\begin{equation}
	\dens(\tm)= \frac{\densinit}{1 + \densinit\recomb\tm},
\end{equation}
kde $\densinit$ je opět počáteční koncentrace částic.

\section{Měření}
\subsection{Experimentální uspořádání}
Schéma aparatury je na obrázku č.~\ref{fig:setup}.
Základem byla výbojka umístěná do středu válcového rezonátoru,
který sloužil k~měření koncentrace elektronů.
Čerpání zajištovala rotační a difuzní vývěva.
Pracovním plynem byl argon, jehož průtok byl regulován jehovým ventilem.
Tlak byl měřen Piraniho manometrem kalibrovaným pro vzduch
a~přepočítáván na argon.
Poloměr výbojky činil \SI{9}{\milli\metre}.

Výboj byl napájen zdrojem vysokého napětí, které bylo přerušováno pomocí
elektronického pentodového spínače.
Spínač byl řízen generátorem pravoúhlých pulzů, který zároveň řídil osciloskop.

Rezonátor tvořila válcová komora o poloměru \SI{40}{\milli\metre}.
Do ní byl přiveden signál z~vysoko\-frekven\-čního laditelného zdroje,
který procházel přes útlumový člen a~přesný vlnoměr.
Procházející signál byl usměrněn diodou a~měřen osciloskopem.

\begin{figure}[hb]
	\ctikzset{bipoles/oscope/waveform=sin}
	\begin{circuitikz}
		% Tube
		\draw (-0.4,0.4) arc (180:360:0.4)
			-- (0.4,5.6)
			arc (0:180:0.4)
			-- (-0.4, 0.4);
		\draw[thick] (0,0) -- (0,0.5);
		\draw[thick] (-0.2,0.5) -- (0.2,0.5);
		\draw[thick] (0,6) -- (0,5.5);
		\draw[thick] (-0.2,5.5) -- (0.2,5.5);

		% Resonator
		\node[rotate=90] at (0.7,3) {rezonátor};
		\draw (0.45,2) -- (1,2) -- (1,4) -- (0.45,4);
		\draw (-0.45,2) -- (-1,2) -- (-1,4) -- (-0.45,4);

		% Resonator source
		\node[vsourcesinshape, rotate=90] (hf) at (-9,3) {};
		\node[jump crossing] (cr) at (-2,3) {};
		\draw (hf) to[vR] (-6,3)
			to[rmeter, t=$\lambda$, label=vlnoměr] (-5,3) to[vR] (cr.w);
		\draw (cr.e) -- (-1,3);

		% Resonator to oscilloscope
		\node[fourport] (osc) at (4,2.6) {osciloskop};
		\node[oscopeshape] at (4,2.6) {};
		\draw (1,0 |- osc.port4) to[empty diode] (osc.port4);

		% Main circuit
		\node[fourport, t=VN] (hv) at (-8.5,0) {};
		\node[fourport, t=M] (m) at (-6,0) {};
		\draw (hv.port2) -- (m.port1);
		\draw (hv.port3) -- (m.port4);
		\draw (m.port3) to[rmeter, t={A}] (m.port3 -| -2,0)
			-- (-2,6.5) -- (0,6.5) -- (0,6);
		\draw (m.port2) to[R, -*, label=$R$] (m.port2 -| 0,0)
			-- (m.port2 -| 2,0) -- (2,0 |- osc.port1) -- (osc.port1);
		\draw (m.port2 -| 0,0) -- (0,0);

		% Trigger
		\node[fourport, t=GM] (gm) at (-6,-2.5) {};
		\draw (m.south) -- (gm.north);
		\draw (gm.east) -- (gm.east -| osc.south) -- (osc.south);
	\end{circuitikz}
	\caption{Schéma měřicí aparatury.
		VN~-- zdroj vysokého napětí;
		M~-- elektronický pentodový spínač;
		GM~-- generátor pravoúhlých pulzů.}
	\label{fig:setup}
\end{figure}

\subsection{Stanovení koncentrace elektronů}
Koncentrace elektronů $\dens$ byla určena rezonátorovou metodou,
kterou popsal J. C. Slater v~roce~1946.
Zaplnění části rezonátoru plazmatem způsobuje změnu jeho rezonanční
frekvence z~původní $\freqbase$ na $\freqplas$.
Označme rozdíl frekvencí $\deltafreq = \freqplas - \freqbase$.
Předpokládáme-li, že v~rezonátoru převládá vid $\mathrm{TM}_{010}$,
lze koncentraci $\dens$ přibližně vyjádřit pomocí poruchové metody jako:
\begin{equation}
	\label{eq:resonator}
	\dens=\frac{\num{0.271}}{\num{0.64}} \frac{\resradius^2}{\tuberadius^2}
		\frac{8\pi^2 \permitvac\emass\freq}{\elemcharge^2} \deltafreq,
\end{equation}
kde $\tuberadius$ je poloměr výbojové trubice, $\emass$ hmotnost elektronu,
$\permitvac$ permitivita vakua, $\elemcharge$ elementární náboj
a~$\freq$ rezonanční frekvence rezonátoru.

Frekvence vstupního signálu $\freqsig$ byla postupně měněna v~rozsahu
od~$\freqbase$ do~$\freqplas$.
Na osciloskopu byl pro každou hodnotu $\freqsig$ odečten čas $\tm$ od vypnutí
proudu ve výbojce, při němž měl výstupní signál maximum.
Pro~tento čas byla rezonátoru přisouzena rezonanční frekvence $\freqsig$
a z~ní spočtena koncentrace částic $\dens$ podle vztahu \eqref{eq:resonator}.
% Takto naměřené hodnoty frekvence jsou na obrázku č.~\ref{fig:freq}.

% \begin{figure}[htp]
% 	\centering
% 	\input{plots/freq}
% 	\caption{Naměřený průběh rezonanční frekvence rezonátoru $\freq$
% 		pro několik hodnot tlaku.}
% 	\label{fig:freq}
% \end{figure}

\subsection{Určení převládajícího procesu}
Koeficienty $\diffuse$ a~$\recomb$ byly určeny nejprve předběžně pomocí
zjednodušených modelů, které předpokládaly vždy jen jeden rozpadový proces.
Uvažujeme-li pouze difuzi, ze vztahu \eqref{eq:dens-diffuse} plyne,
že funkce $\ln(\dens)$ má být lineární.
Naměřená data byla aproximována touto přímkou a~z její směrnice
byl spočten první odhad koeficientu difuze $\diffuse$.
Pro objemovou rekombinaci platí podle vztahu \eqref{eq:dens-recomb},
že závislost $1/\dens$ má být lineární.
Z~této aproximace byl určen počáteční odhad koeficientu rekombinace $\recomb$.
Hodnoty $\ln(\dens)$, $1/\dens$, jimi proložené přímky a~z~nich spočtené
parametry jsou na obrázcích č.~\ref{fig:fit1} a~\ref{fig:fit2}.

\newcommand\plotfit[1]{%
	\sisetup{
		round-mode = figures,
		round-precision = 2,
	}%
	\input{plots/fit-log-#1}%
	\input{plots/fit-rec-#1}%
}
\begin{figure}[htp]
	\centering
	\plotfit{10}
	\plotfit{20}
	\plotfit{50}
	\plotfit{100}%
	\caption{Určování koeficientu difuze $\diffuse$ a~rekombinace $\recomb$.}
	\label{fig:fit1}
\end{figure}
\begin{figure}[htp]
	\centering
	\plotfit{200}
	\plotfit{300}
	\plotfit{380}%
	\caption{Určování koeficientu difuze $\diffuse$ a~rekombinace $\recomb$.}
	\label{fig:fit2}
\end{figure}

\newcommand\paramc{C}
Pro přesnější stanovení obou koeficientů je nutno použít model,
který zahrnuje oba procesy. Tento je popsán diferenciální rovnicí:
\begin{equation}
	\label{eq:dens-general-eq}
	\frac{\partial \dens(\tm)}{\partial \tm}
		= \frac{\diffuse}{\diffuselen^2}\dens(\tm) - \recomb\dens^2(\tm),
\end{equation}
jejíž řešení lze napsat ve tvaru:
\begin{equation}
	\label{eq:dens-general}
	\dens(\tm)=\frac{1}{
		\paramc\euler^{\tm\diffuse/\diffuselen^2}
		- \frac{\recomb\diffuselen^2}{\diffuse}
	}.
\end{equation}
Aproximací dat touto funkcí získáme kromě koeficientů $\diffuse$ a $\recomb$
také konstantu $\paramc$.

Proklad byl proveden Levenberg-Marquardtovým algoritmem.
Počáteční hodnoty parametrů byly určeny z~předběžně určených hodnot
koeficientů.
Koncentrace částic v~plazmatu v~době trvání výboje $\densinit$
byla extrapolována dosazením $\tm = 0$ do proložené funkce.
Spočtené průběhy koncentrací $\dens$ a~přísluš\-ných modelových funkcí
jsou vyneseny na~obrázku č.~\ref{fig:density}.
Pro porovnání s~předběžnými proklady jsou uvedeny i~v~reciproké
a~logaritmické podobě,
což viz na obrázcích č.~\ref{fig:density-inv} a~\ref{fig:density-log}.

\begin{figure}[htp]
	\centering
	\input{plots/density}
	\caption{Naměřená koncentrace částic $\dens$
		a její aproximace modelovou funkcí \eqref{eq:dens-general}.}
	\label{fig:density}
\end{figure}
\begin{figure}[htp]
	\centering
	\input{plots/density-inv}
	\caption{Naměřená koncentrace částic $\dens$
		a její aproximace modelovou funkcí \eqref{eq:dens-general}
		(převrácená hodnota).}
	\label{fig:density-inv}
\end{figure}
\begin{figure}[htp]
	\centering
	\input{plots/density-log}
	\caption{Naměřená koncentrace částic $\dens$
		a její aproximace modelovou funkcí \eqref{eq:dens-general}
		(logaritmus).}
	\label{fig:density-log}
\end{figure}

\clearpage
\section{Výsledky}
Měření bylo opakováno pro sedm hodnot tlaku.
Výsledky jsou shrnuty v~tabulce č.~\ref{tab:results}
spolu s~nejistotami určenými z~nalezených parametrů modelu.

V~grafech na obrázku č.~\ref{fig:diffuse-recomb-pres} je naměřená
závislost koeficientů $\diffuse$ a~$\recomb$ na tlaku $\pres$.
Je patrno, že koeficient $\diffuse$ klesá s~rostoucím tlakem.
To je v~souladu s~očekáváním, že difuze se bude více projevovat
při nižších tlacích.
Překvapivým výsledkem je průběh koeficientu $\recomb$.
Ve~vyšších tlacích roste, což odpovídá očekávání,
ale při hodnotách $\SIlist{10;20}{\pascal}$ je neúměrně vysoký.
Tuto anomálii se nepodařilo objasnit.

\shorthandoff{-}
\begin{table}[hb]
	\centering
	\caption{Hodnoty difuzního koeficientu $\diffuse$,
		koeficientu rekombinace $\recomb$ a koncentrace elektronů $\dens$
		určené různými metodami.}
	\label{tab:results}
	\pgfplotstableset{
		header = false,
		skip first n = 1,
		col sep = tab,
	}
	\sisetup{
		round-mode = uncertainty,
		round-precision = 1,
		separate-uncertainty = true,
		table-alignment-mode = format,
		table-number-alignment = center,
	}
	\pgfplotstabletypeset[
		string type,
		columns = {[index] 0, [index] 2, [index] 3, [index] 4,
			[index] 5, [index] 6, [index] 7, [index] 8},
		column type = {@{}
			S[table-format = 3]
			S[table-format = 3.0(1)]
			S[table-format = 2.1(1)]
			S[table-format = 3.1(1)]
			S[table-format = +4.0(4)]
			S[table-format = 3.0(2)]
			S[table-format = 2.0(2)]
			S[table-format = 3.0(3)]
		@{}},
		every head row/.append style = {
			output empty row,
			before row = {
				\toprule
				&
				\multicolumn{2}{c}{$\ln\dens$} &
				\multicolumn{2}{c}{$1/\dens$} &
				\multicolumn{3}{c}{obecné $\dens$}
				\\
				\cmidrule(lr){2-3}
				\cmidrule(lr){4-5}
				\cmidrule(lr){6-8}
				\multicolumn{1}{c}{$\pres$} &
				\multicolumn{1}{c}{$\diffuse$} &
				\multicolumn{1}{c}{$\densinit$} &
				\multicolumn{1}{c}{$\recomb$} &
				\multicolumn{1}{c}{$\densinit$} &
				\multicolumn{1}{c}{$\diffuse$} &
				\multicolumn{1}{c}{$\recomb$} &
				\multicolumn{1}{c}{$\densinit$}
				\\
				$[\si{\pascal}]$ &
				$[\si{\diffuseunit}]$ &
				$[\SI{e15}{\metre^{-3}}]$ &
				$[\SI{e-20}{\recombunit}]$ &
				$[\SI{e15}{\metre^{-3}}]$ &
				$[\si{\diffuseunit}]$ &
				$[\SI{e-20}{\recombunit}]$ &
				$[\SI{e15}{\metre^{-3}}]$
				\\
			},
		},
		zerofill,
	]{results/summary.tsv}
\end{table}
\shorthandon{-}

\begin{figure}[htp]
	\centering
	\input{plots/diffuse-pres}
	\input{plots/recomb-pres}
	\caption{Závislost koeficientu difuze $\diffuse$
		a~koeficientu rekombinace $\recomb$ na tlaku $\pres$.}
	\label{fig:diffuse-recomb-pres}
\end{figure}

% \begin{figure}[htp]
% 	\centering
% 	\input{plots/density-pres}
% 	\caption{Závislost koncentrace částic $\densinit$ na tlaku $\pres$.}
% 	\label{fig:density-pres}
% \end{figure}

\section{Závěr}
Byly pozorovány a~vyhodnocovány dva mechanismy rozpadu plazmatu.
Prvním z~nich byla difuze elektronů na stěny,
druhým byla rekombinace částic v~objemu plazmatu.
Z~časového vývoje koncentrace částic $\dens$ při dohasínání výboje
byl určen koeficient difuze $\diffuse$
a~koeficient objemové rekombinace $\recomb$.

Ukázalo se, že difuzní koeficient $\diffuse$ vycházel řádově
v~desítkách $\si{\diffuseunit}$ a~s~rostoucím tlakem klesal.
Koeficient rekombinace $\recomb$ vycházel v~řádu $\SI{e-20}{\recombunit}$
a~s~rostoucím tlakem rostl.

\end{document}
