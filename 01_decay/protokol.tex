\documentclass{protokol}

\usepackage[czech]{babel}
\usepackage[utf8]{inputenc}
\usepackage{icomma}

% Plovouci bloky (obrazky, tabulky)
\usepackage{floatrow}
\floatsetup[table]{capposition=top}
\floatsetup[figure]{frameset={\fboxsep16pt}}
\usepackage{subcaption}

% Tabulky
\usepackage{tabu}
\usepackage{booktabs}
\usepackage{csvsimple}
\usepackage{multirow}
\usepackage{multicol}
\usepackage{pgfplotstable}
\pgfplotstableset{
	use comma,
	set thousands separator = {},
	every head row/.style = {
		before row = \toprule,
		after row = \midrule
	},
	every last row/.style = {after row = \bottomrule},
}

% Jednotky
\usepackage{siunitx}
\sisetup{
	locale               = DE,
	group-digits         = false,
	inter-unit-product   = \ensuremath{{}\cdot{}},
	list-units           = single,
	list-separator       = {; },
	list-final-separator = \text{ a },
	list-pair-separator  = \text{ a },
	range-phrase         = \text{ až },
	range-units          = single,
	per-mode             = symbol,
}
\usepackage{amsmath}

% Obvody
\usepackage{circuitikz}

% Obrazky a grafy
% \usepackage{graphicx}
\graphicspath{
	{plots/}
	{build/plots/}
}
\usepackage{epstopdf}
\epstopdfsetup{outdir=./build/plots/}
\pgfplotsset{compat=1.17}

\jmenopraktika={Praktikum z fyziky plazmatu}  % jmeno predmetu
\jmeno={Pavel Kosík, Jan Slaný}                             % jmeno mericiho
\obor={F}                               % zkratka studovaneho oboru
\skupina={Út 15:00}                     % doba vyuky seminarni skupiny
\rocnik={IV}
\semestr={VIII}

\cisloulohy={01}
\jmenoulohy={Rozpad plazmatu}

\datum={1. března 2022}                  % datum mereni ulohy
\tlak={}% [hPa]
\teplota={}% [C]
\vlhkost={}% [%]

\newcommand\euler{e}
\newcommand\emass{m_\mathrm{e}}
\newcommand\elemcharge{e}
\newcommand\permitvac{\varepsilon_0}

\newcommand\tm{t}
\newcommand\freq{f}
\newcommand\freqbase{\freq_0}
\newcommand\freqplas{\freq_1}
\newcommand\freqsig{\freq'}
\newcommand\deltafreq{\Delta\freq}
\newcommand\curdis{I}
\newcommand\dens{n}
\newcommand\densinit{n_0}
\newcommand\pres{p}
\newcommand\diffuse{D}
\newcommand\diffuseunit{\metre\squared\per\second}
\newcommand\recomb{\alpha}
\newcommand\recombunit{\metre\cubed\per\second}
\newcommand\diffuselen{\Lambda}
\newcommand\doll{\diffuse/\diffuselen^2}
\newcommand\dollunit{\second^{-1}}
\newcommand\tuberadius{r_\mathrm{t}}
\newcommand\resradius{r_\mathrm{r}}

\begin{document}
\header

\section{Teorie}
Po přerušení přívodu energie do plazmatu se plazma začne rozpadat.
Rozpadem rozumíme postupné klesání koncentrace nabitých částic,
které je způsobeno převážně objemovou rekombinací a~difuzí,
při níž elektrony difundují na stěny výbojky a rekombinují.

Úkolem tohoto praktika je zkoumat konkrétní výboj a~určit,
který z~procesů při jeho rozpadu převládá.
K~tomu je nutno stanovit koeficient difuze $\diffuse$
a~koeficient objemové rekombinace $\recomb$.

\subsection{Difuze v plazmatu}
Při nižších tlacích je převládajícím mechanismem rozpadu difuze.
Pro popis difuzních ztrát vyjdeme z~rovnice kontinuity elektronů:
\begin{equation}
	\label{eq:continuity}
	\frac{\partial \dens}{\partial \tm} - \nabla^2 (\diffuse\dens) = 0,
\end{equation}
kde $\dens$ je koncentrace elektronů a $\diffuse$ je koeficient difuze.
Za předpokladu, že $\dens$ je konstantní v~celém objemu,
výbojka má válcový tvar a že difuze probíhá výhradně v~základním vidu,
můžeme koncentraci elektronů v~čase vyjádřit jako:
\begin{equation}
	\label{eq:dens-diffuse}
	\dens(\tm)= \densinit\euler^{-\frac{\diffuse}{\diffuselen^2}\tm},
\end{equation}
kde $\densinit = \dens(0)$ je počáteční koncentrace
a~$\diffuselen$ je difuzní délka.
Její hodnota je přibližně $\diffuselen \approx \frac{\tuberadius}{\num{2.405}}$,
kde $\tuberadius$ je poloměr trubice a~faktor $\num{2.405}$ je první kořen
Besselovy funkce prvního druhu $J_0$.

\subsection{Objemová rekombinace}
S~rostoucím tlakem roste vliv rekombinace nabitých částic v~objemu plazmatu.
Rekombinační ztráty lze popsat jednoduchou diferenciální rovnicí:
\begin{equation}
	\label{eq:dens-recomb}
	\frac{\mathrm{d} n}{\mathrm{d} t} = -\recomb \dens^2,
\end{equation}
kde $\recomb$ je koeficient rekombinace.
Řešením této rovnice je časový vývoj ve~tvaru:
\begin{equation}
	\dens(\tm)= \frac{\densinit}{1 + \densinit\recomb\tm},
\end{equation}
kde $\densinit$ je opět počáteční koncentrace částic.

\section{Měření}

\subsection{Stanovení koncentrace elektronů}
Časový průběh koncentrace elektronů $\dens$ při dohasínání výboje
byl určen pomocí rezonátorové metody popsané J. C. Slaterem v~roce 1946.
Výbojová trubice byla umístěna do středu válcového rezonátoru,
do něhož byl přiveden laditelný vysokofrekvenční signál.
Výstupní signál byl snímán diodou připojenou na osciloskop.
Zaplnění části rezonátoru plazmatem způsobuje změnu jeho rezonanční
frekvence z~původní $\freqbase$ na $\freqplas$.
Označme rozdíl frekvencí $\deltafreq = \freqplas - \freqbase$.
Předpokládáme-li, že v~rezonátoru převládá vid $\mathrm{TM}_{010}$,
lze koncentraci $\dens$ přibližně vyjádřit pomocí poruchové metody jako:
\begin{equation}
	\label{eq:resonator}
	\dens=\frac{\num{0.271}}{\num{0.64}} \frac{\resradius^2}{\tuberadius^2}
		\frac{8\pi^2 \permitvac\emass\freq}{\elemcharge^2} \deltafreq,
\end{equation}
kde $\tuberadius$ je poloměr výbojové trubice, $\emass$ hmotnost elektronu,
$\permitvac$ permitivita vakua, $\elemcharge$ elementární náboj
a~$\freq$ rezonanční frekvence rezonátoru.

Frekvence vstupního signálu $\freqsig$ byla postupně měněna v~rozsahu
od~$\freqbase$ do~$\freqplas$.
Na osciloskopu byl pro každou hodnotu $\freqsig$ odečten čas $\tm$ od vypnutí
proudu ve výbojce, při němž měl výstupní signál maximum.
Pro~tento čas byla rezonátoru přisouzena rezonanční frekvence $\freqsig$
a z~ní spočtena koncentrace částic $\dens$ podle vztahu \eqref{eq:resonator}.
Takto naměřené hodnoty frekvence jsou na obrázku č.~\ref{fig:freq}.

\begin{figure}[htp]
	\centering
	\input{plots/freq}
	\caption{Naměřený průběh rezonanční frekvence rezonátoru $\freq$
		pro několik hodnot tlaku.}
	\label{fig:freq}
\end{figure}

\subsection{Určení převládajícího procesu}
Koeficienty $\diffuse$ a~$\recomb$ byly určeny nejprve předběžně pomocí
zjednodušených modelů, které předpokládaly vždy jen jeden rozpadový proces.
Uvažujeme-li pouze difuzi, ze vztahu \eqref{eq:dens-diffuse} plyne,
že funkce $\ln(\dens)$ má být lineární.
Naměřená data byla aproximována touto přímkou a~z její směrnice
byl spočten první odhad koeficientu difuze $\diffuse$.
Pro objemovou rekombinaci platí podle vztahu \eqref{eq:dens-recomb},
že závislost $1/\dens$ má být lineární.
Z~této aproximace byl určen počáteční odhad koeficientu rekombinace $\recomb$.
Hodnoty $\ln(\dens)$, $1/\dens$, jimi proložené přímky a~z~nich spočtené
parametry jsou na obrázcích č.~\ref{fig:fit1} a~\ref{fig:fit2}.

\newcommand\plotfit[1]{%
	\sisetup{
		round-mode = figures,
		round-precision = 2,
	}%
	\input{plots/fit-log-#1}%
	\input{plots/fit-rec-#1}%
}
\begin{figure}[htp]
	\centering
	\plotfit{10}
	\plotfit{20}
	\plotfit{50}
	\plotfit{100}%
	\caption{Určování koeficientu difuze $\diffuse$ a~rekombinace $\recomb$.}
	\label{fig:fit1}
\end{figure}
\begin{figure}[htp]
	\centering
	\plotfit{200}
	\plotfit{300}
	\plotfit{380}%
	\caption{Určování koeficientu difuze $\diffuse$ a~rekombinace $\recomb$.}
	\label{fig:fit2}
\end{figure}

\newcommand\paramc{C}
Pro přesnější stanovení obou koeficientů je nutno použít model,
který zahrnuje oba procesy. Tento je popsán diferenciální rovnicí:
\begin{equation}
	\label{eq:dens-general-eq}
	\frac{\partial \dens(\tm)}{\partial \tm}
		= \frac{\diffuse}{\diffuselen^2}\dens(\tm) - \recomb\dens^2(\tm),
\end{equation}
jejíž řešení lze napsat ve tvaru:
\begin{equation}
	\label{eq:dens-general}
	\dens(\tm)=\frac{1}{
		\paramc\euler^{\tm\diffuse/\diffuselen^2}
		- \frac{\recomb\diffuselen^2}{\diffuse}
	}.
\end{equation}
Aproximací dat touto funkcí získáme kromě koeficientů $\diffuse$ a $\recomb$
také konstantu $\paramc$.

Proklad byl proveden Levenberg-Marquardtovým algoritmem.
Počáteční hodnoty parametrů byly určeny z~předběžně určených hodnot
koeficientů.
Koncentrace částic v~plazmatu v~době trvání výboje $\densinit$
byla extrapolována dosazením $\tm = 0$ do proložené funkce.
Spočtené průběhy koncentrací $\dens$ a~přísluš\-ných modelových funkcí
jsou vyneseny na~obrázku č.~\ref{fig:density}.

\begin{figure}[htp]
	\centering
	\input{plots/density}
	\caption{Naměřená koncentrace částic $\dens$
		a její aproximace modelovou funkcí \eqref{eq:dens-general}.}
	\label{fig:density}
\end{figure}
\begin{figure}[htp]
	\centering
	\input{plots/density-inv}
	\caption{Naměřená koncentrace částic $\dens$
		a její aproximace modelovou funkcí \eqref{eq:dens-general}
		(převrácená hodnota).}
	\label{fig:density-inv}
\end{figure}
\begin{figure}[htp]
	\centering
	\input{plots/density-log}
	\caption{Naměřená koncentrace částic $\dens$
		a její aproximace modelovou funkcí \eqref{eq:dens-general}
		(logaritmus).}
	\label{fig:density-log}
\end{figure}

\section{Výsledky}
Měření bylo opakováno pro sedm hodnot tlaku.
V~grafech na obrázku č.~\ref{fig:diffuse-recomb-pres} je naměřená
závislost koeficientů $\diffuse$ a~$\recomb$ na tlaku $\pres$.
Je patrno, že koeficient $\diffuse$ klesá s~rostoucím tlakem.
To je v~souladu s~očekáváním, že difuze se bude více projevovat
při nižších tlacích.
Překvapivým výsledkem je průběh koeficientu $\recomb$.
Ve~vyšších tlacích roste, což odpovídá očekávání,
ale při hodnotách $\SIlist{10;20}{\pascal}$ je neúměrně vysoký.
Tuto anomálii se nepodařilo objasnit.

\shorthandoff{-}
\begin{table}
	\centering
	\caption{Hodnoty difuzního koeficientu $\diffuse$,
		koeficientu rekombinace $\recomb$ a koncentrace elektronů $\dens$
		určené různými metodami.}
	\label{tab:results}
	\pgfplotstableset{
		header = false,
		skip first n = 1,
		col sep = tab,
	}
	\pgfplotstabletypeset[
		columns = {[index] 0, [index] 2, [index] 3, [index] 4,
			[index] 5, [index] 6, [index] 7, [index] 8},
		column type = {r r r r r r r r},
		every head row/.append style = {
			output empty row,
			before row = {
				\toprule
				&
				\multicolumn{2}{c}{$\ln\dens$} &
				\multicolumn{2}{c}{$1/\dens$} &
				\multicolumn{3}{c}{obecné $\dens$}
				\\
				\cmidrule(lr){2-3}
				\cmidrule(lr){4-5}
				\cmidrule(lr){6-8}
				\multicolumn{1}{c}{$\pres$} &
				\multicolumn{1}{c}{$\diffuse$} &
				\multicolumn{1}{c}{$\densinit$} &
				\multicolumn{1}{c}{$\recomb$} &
				\multicolumn{1}{c}{$\densinit$} &
				\multicolumn{1}{c}{$\diffuse$} &
				\multicolumn{1}{c}{$\recomb$} &
				\multicolumn{1}{c}{$\densinit$}
				\\
				$[\si{\pascal}]$ &
				$[\si{\diffuseunit}]$ &
				$[\SI{e15}{\metre^{-3}}]$ &
				$[\SI{e-20}{\recombunit}]$ &
				$[\SI{e15}{\metre^{-3}}]$ &
				$[\si{\diffuseunit}]$ &
				$[\SI{e-20}{\recombunit}]$ &
				$[\SI{e15}{\metre^{-3}}]$
				\\
			},
		},
		zerofill,
		columns/0/.style = {
			column name = $\pres\,[\si{\pascal}]$,
			precision = 0,
		},
		columns/2/.style = {
			column name = $\diffuse\,[\si{\metre\squared\per\second}]$,
			precision = 1,
		},
		columns/3/.style = {
			column name = $\densinit\,[\SI{e15}{\metre^{-3}}]$,
			precision = 1,
			multiply with = 1e-15,
		},
		columns/4/.style = {
			column name = $\recomb\,[\si{\metre\squared\per\second}]$,
			precision = 3,
			multiply with = 1e18,
		},
		columns/5/.style = {
			column name = $\densinit\,[\SI{e15}{\metre^{-3}}]$,
			precision = 1,
			fixed,
			multiply with = 1e-15,
		},
		columns/6/.style = {
			column name = $\diffuse\,[\si{\metre\squared\per\second}]$,
			precision = 1,
		},
		columns/7/.style = {
			column name = $\recomb\,[\SI{e-20}{\metre\cubed\per\second}]$,
			fixed,
			precision = 3,
			multiply with = 1e18,
		},
		columns/8/.style = {
			column name = $\densinit\,[\SI{e15}{\metre^{-3}}]$,
			precision = 1,
			multiply with = 1e-15,
		},
	]{results/summary.tsv}
\end{table}
\shorthandon{-}

\begin{figure}[htp]
	\centering
	\input{plots/diffuse-pres}
	\input{plots/recomb-pres}
	\caption{Závislost koeficientu difuze $\diffuse$
		a~koeficientu rekombinace $\recomb$ na tlaku $\pres$.}
	\label{fig:diffuse-recomb-pres}
\end{figure}

\begin{figure}[htp]
	\centering
	\input{plots/density-pres}
	\caption{Závislost koncentrace částic $\densinit$ na tlaku $\pres$.}
	\label{fig:density-pres}
\end{figure}

\end{document}
